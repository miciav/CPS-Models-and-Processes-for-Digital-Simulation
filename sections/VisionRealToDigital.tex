In questa sezione raccogliamo un insieme di considerazioni che riguardano il futuro (se non immediato almeno prossimo) della simulazione applicata agli ambiti industriali. 
In particolare, una specifica visione di un possibile impatto del digital sul reale (Digital-to-Real) viene presentata. Infatti, essendo l'obiettivo di questo capitolo quello di documentare la fattibilità di una piattaforma in grado di gestire i gemelli digitali in maniera che questi rispecchino fedelmente e continuamente lo stato della loro controparte reale, abbiamo lasciato poco spazio al processo decisionale conseguente alla simulazione. Questo perché al momento è ancora molto macchinoso agire sul sistema reale per implementare le intuizioni ottenute per mezzo della simulazione. 
Ciò nonostante almeno in due scenari è possibile immaginare di chiudere il ciclo e agire direttamente sul reale. 
Il primo scenario riguarda la reazione ad eventi inaspettati, come l'arrivo di un ordine particolarmente importante o la rottura di un macchinario, in casi come questi la simulazione può essere molto utile per reagire all'evento ottenendo in tempi brevi configurazioni alternative. Siccome già ora in molti casi la produzione è gestita per mezzo di applicativi software MES/ERP, possiamo immaginare un'integrazione tra i meccanismi di pianificazione e gestione della produzione e la simulazione.
Il secondo scenario invece riguarda il lato Cyber dei CPS. Già ora il rapporto tra software dei CPS e strumenti di simulazione è molto stretto. La realtà simulata permette infatti di testare il comportamento di un macchinario prima di inserirlo effettivamente all'interno di un processo produttivo. Questo approccio ha il vantaggio di accorciare i tempi di sviluppo e migliorare la sicurezza nella fabbrica.  