This Section collects a number of considerations accrued in the frame of this research about the future (if not immediate at least near) of simulation applied to industrial environments. 
We refer with the phrase \textit{simulation-in-the-loop}, the set of processes that realize the virtual factory (real-to-digital) and those that enable the application of the results of the simulation to the real factory (digital-to-real). 
In fact, since the objective of this Chapter is to document the feasibility of a platform capable of managing digital twins so that they faithfully and continuously reflect the state of their real counterpart, it has been intentionally left little room for the downstream decision-making process of the simulation. 
This is because at the moment, in many contexts, it can be still cumbersome to operate on the real system to implement the insights obtained through simulation. 
Nevertheless, at least in two scenarios it is possible to imagine to close the loop and act directly on the processes of the real factory. 

The first scenario concerns the reaction to unexpected events, such as the arrival of a particularly important order or a machine breakdown; in cases like these the simulation can be very useful to react swiftly by generating alternative configurations. Since the production is already managed digitally by means of MES/ERP software applications, we can imagine a full integration between the mechanisms of planning and management of the production and the simulation to achieve a flexible and reactive manufacturing. 

In the second scenario we can imagine to interact with the cyber nature of a CPS. The relationship between software run by CPS and simulation tools is already very close. The simulated reality allows in fact to test the behavior of a machine before actually placing it in a production line. This approach has the advantage of shortening development times and improving factory safety. Once production has started, however, during the CPS life cycle it may be necessary to update the software for various reasons. Therefore, one can picture a system that, being able to take advantage of the constantly updated digital twin, can automatically adjust the settings of the software in order to make it always work in optimal conditions or even improve its performance (speed of operation, power consumption, and so on). The new code can, after being validated by a human expert, be uploaded directly to the CPS reducing to a minimum the downtime of the line. 
