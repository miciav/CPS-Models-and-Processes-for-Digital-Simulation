This section collects a number of reflections on the future (if not immediate at least near) of simulation applied to industrial environments, what we call "simulation-in-the-loop", that is, the set of processes that realize the virtual factory (real-to-digital) and those that enable the application of the results of the simulation to the real factory (digital-to-real). 
In fact, since the objective of this chapter is to document the feasibility of a platform capable of managing digital twins so that they faithfully and continuously reflect the state of their real counterpart, has been intentionally left little room for the decision-making process resulting from the simulation. 
This is because at the moment it is still very cumbersome to operate on the real system to implement the insights obtained through simulation. 
Nevertheless, at least in two scenarios it is possible to imagine to close the loop and act directly on the processes of the real factory. 
The first scenario concerns the reaction to unexpected events, such as the arrival of a particularly important order or a machine breakdown; in cases like these the simulation can be very useful to react to the event obtaining in a short time alternative configurations. Since the production is already managed digitally in many cases by means of MES/ERP software applications, we can imagine an integration between the mechanisms of planning and management of the production and the simulation to obtain a flexible and reactive manufacturing. 
In the second scenario let us consider the cyber nature of a CPS. The relationship between software run by CPS and simulation tools is already very close. The simulated reality allows in fact to test the behavior of a machine before actually placing it in a production line. This approach has the advantage of shortening development times and improving factory safety. Once production has started, however, in the remaining CPS life cycle, it may be necessary to update the software for various reasons. Therefore, one can imagine a system that, being able to take advantage of the constantly updated digital twin, can automatically adjust the settings of the real CPS software in order to make it always work in optimal conditions or even improve its performance (speed of operations, power consumption, and so on). The new code can, after being validated by a human expert, be uploaded directly to the inside of the CPS reducing to a minimum the downtime of the line. 
