
This Chapter reported on recent research activities on CPS virtualization and real-to-digital synchronization, aimed mainly at conceptualizing the synchronization and triggering processes, and at developing a prototype of the real-to-digital synchronization middleware.
In particular, the work carried out focused on the specification of a CPS virtualization and simulation model, and on different modes for the synchronization of the factory model with the real production environment considering that, on the one hand, there is a low-intensity discovery process of the smart elements that are deployed, activated, connected and disconnected during the factory activities that requires the virtual model to be automatically updated to hold continuously a coherent representation of the real factory. 
On the other hand, there is a more regular flow of information related to the production activities and, in particular, tracing the flow of manufactured product units during the logistics process. This calls for different data handling and synchronization approaches that specifically address the two issues. In this chapter the second one has been addressed by analyzing the problem, conceiving the synchronization processes and developing the necessary software components.

As far as the virtualization and simulation model is concerned, this has been expressly designed to support the real-to-digital synchronization working in two complementary directions:
\begin{enumerate}
    \item the creation of a meta-model as a baseline to define use-case specific modellings. In particular the model enforces the distinction between transparent items, which both human and machine readable, and black-box elements, which represents proprietary or solution specific artifacts. This distinction is particularly important to foster the adoption of the model among different vendors;
    \item the definition of specific features in the models appointed to enable the synchronization. In particular, \textit{triggers} and \textit{Synchronization Models} are the tool made available to the modeler to manage events and the continuous updating of digital twins, respectively. 
\end{enumerate}


The future research will mainly focus on \textit{closing the loop} implementing strategies and technologies for simulation-based configuration and reconfiguration of smart elements. In order to do so, we will work on finalizing the integration with the discovery services acting in the Edge Tier and with the Ledger Tier for the synchronization of the data model through the Open APIs for Virtualization.
